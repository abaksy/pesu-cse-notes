\documentclass{article}
\usepackage[utf8]{inputenc}
\usepackage[a4paper, portrait, margin=0.6in]{geometry}
\usepackage{graphicx}
\usepackage{subcaption}
\usepackage{amsmath}
\usepackage{geometry}
\usepackage{array}


\title{Software Testing (UE18CS400SB) \\ Unit 4}
\author{Aronya Baksy}
\date{November 2021}

\begin{document}
\maketitle
\section{Acceptance Testing}
\begin{itemize}
    \item Testing done in accordance with customer specified criteria (mutually agreed)
    
    \item Done in the customer environment by customer-specified individuals
    
    \item Types of Acceptance testing:
    \begin{itemize}
        \item User Acceptance Testing
        
        \item Business Acceptance Testing
        
        \item Contract Acceptance Testing
        
        \item Regulations Acceptance Testing
        
        \item Operational Acceptance Testing
        
        \item $\alpha$ and $\beta$ testing
    \end{itemize}
    
    \item Approaches to acceptance testing:
    \begin{itemize}
        \item \textbf{Design / Architecture based approach}: Full functionality, Cases that fail here may trigger review

        \item \textbf{Business vertical / customized instance}:
Typically for products / frameworks
Customized instances are put through testing based on business domain

        \item \textbf{ Deployment focused}: Large applications tested in complex hw or sw environments
    \end{itemize}
    
    \item Criteria for acceptance testing:
    \begin{itemize}
        \item SPecifying the business requirements (may be functional or non functional)
        
        \item May be process requirements (test coverage, training etc.)
    \end{itemize}
    
    \item Acceptance test cases cover: Critical functionality, End-to-end scenarios. New functionalities – during upgrade, Legal / statutory needs, Functionality to work on a defined corpus
    
    \item Acceptance test execution: on site, in the presence of dev engineers, with proper documentation of the execution process
    
    \item Challenges: criteria not easy to specify, multiple iterations with multiple cust. representatives needed
\end{itemize}

\section{Non-Functional Testing}
\begin{itemize}
    \item Testing the non-functional attributes (performance, reliability, scalability, load, security, compatibility and others) of the system
    
    \item Reasons:
    \begin{itemize}
        \item Identify and correct design faults
        
        \item Identify limits of the system
        
        \item Whether the product can behave gracefully during stress and load conditions

        \item To ensure that the product can work without degrading  for a long duration

        \item To compare with other products and previous versions

        \item To avoid un-intended side effects.
    \end{itemize}
    
    \item When the product has no basic issues and meets the minimum entry criteria, then the non-functional test can start.
    
    \item The non-functional testing is stopped when there is enough data to make a judgement. This must be aligned with release schedule
\end{itemize}

\subsection{Types of Non-Functional Testing}
\subsubsection{Scalability Test}
\begin{itemize}
    \item Ability of a system to handle increasing amounts of  work without unacceptable level of performance  (degradation)

    \item Scalability may be vertical or horizontal
    
    \item The test cases will focus on to test the  maximum limits of the features, utilities and  performing some basic operations
\end{itemize}

\subsubsection{Performance Test}
\begin{itemize}
    \item Evaluating response time of product to perform required actions under  stated conditions 
    
    \item comparison with different versions of same product and competitive products.
    
    \item Test cases focus on getting response time  and throughput  for  different  operations,  under  defined environment and load
\end{itemize}

\subsubsection{Reliability Test}
\begin{itemize}
    \item Evaluate the ability of the product to  perform it's required functions under stated conditions for a  specified period of time or number of iterations
    
    \item The test cases will focus on the product failures  when the operations are executed continuously  for a given duration or iterations
    
    \item Focus on frequently used operations
\end{itemize}
\subsubsection{Stress Test}
\begin{itemize}
    \item Evaluate a system beyond the limits of  the specified requirements or environment resources to ensure the product behavior is acceptable.
    
    \item Well-designed system shows graceful degradation and safe behaviour
    
    \item System should show performance decrease when resource/load ratio reduces, and symmetric increases when load is reduced
\end{itemize}

\subsubsection{Security Test}
\begin{itemize}
    \item Both static and dynamic in nature
    
    \item Test tools identify vulnerabilities at  application level (access control, SQL injection, buffer overflow, encryption, API design)

\end{itemize}

\subsubsection{Regression Test}
\begin{itemize}
    \item A \textbf{black-box} technique, that ensures that code change does not impact existing functionality
    
    \item Types of regression testing:
    \begin{itemize}
        \item \textbf{Corrective}: No changes to requirements, reuse existing test cases
        
        \item \textbf{Retest-all}: Re-doing all tests again, not advisable for minor changes
        
        \item \textbf{Selective}: Using a subset of the test cases, analyze impact of new code on existing code
        
        \item \textbf{Progressive}: When change in spec, new test cases, ensures that new features do not break existing ones
        
        \item \textbf{Complete}: Used for major changes in code. 
        
        \item \textbf{Partial}: Tests issues when new code is added to already existing code, ensures that a system continues to work after adding new code
        
        \item \textbf{Unit}: focus on a single unit, block all dependencies

    \end{itemize}
\end{itemize}
\end{document}