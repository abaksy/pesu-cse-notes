\documentclass{article}
\usepackage[utf8]{inputenc}
\usepackage{graphicx}
\usepackage{hyperref}
\usepackage{multirow}
\usepackage{amsmath}
\usepackage{textcomp}

\title{Research Methodology UE18CS400SG \\ Unit 5}
\author{Aditeya Baral}
\date{April 2022}

\begin{document}

\maketitle

\section{Research Proposal}

\begin{itemize}
    \item Statement of intent on \textbf{what the study is about, why is it important, how it will be conducted and some insights on results}
    \item Concise, coherent summary and plan of proposed work
    \item Submitted to panel after initial studies, use future tense
    \item Helps clear thought process on micro levels -- \textbf{familiarises background, justifies research, helps anticipate timeline, problems, design, experiments}
\end{itemize}

\subsection{Format of Research Proposal}

\begin{enumerate}
    \item \textbf{Literature Review}
    \item \textbf{Aim or Objectives}
    \item \textbf{Proposed Methodology}-- justify methods, prove feasibility
    \item \textbf{Proposed Experimental Design}
    \item \textbf{Timeline} -- helps keep design in check, avoids dead time, consult others with work in same field
    \item \textbf{Budget (if applicable)} -- appreciation for research costs, prevents overspending, add explanations for each requirement
\end{enumerate}

\section{Ethics}

\begin{itemize}
    \item Research based on trust, be honest
    \item Follow \textbf{Code of Ethics -- IEEE, ACM, 10 Commandments of Computer Ethics}
    \item \textbf{Authenticity and Accuracy} -- report data for which experiments are carried out
    \item \textbf{Originality} -- present your data only, give credits to others
    \item While citing, \textbf{indicate and quote exactly what material is taken from source} reference
    \item \textbf{No plagiarism} -- no self-plagiarism either
    \item \textbf{Accurate, reproducible results} -- include issues, drawbacks, limitations, unexpected and conflicting outcomes
    \item Provide \textbf{authorship based on contribution} to work and content
    \item Ethical treatment of humans and animals
    \item \textbf{Disclosure of conflicts of interest} -- disclose commercial relations, grant proposals, opt for blind reviews (reviewers unknown to authors, authorship removed from paper before review)
\end{itemize}

\section{Plagiarism}

\textit{Appropriation of another person’s ideas, processes, results, words, figures, illustrations without giving appropriate credit.}

\begin{itemize}
    \item Ethical issue and should be avoided
    \item Software exist to detect parts of a document with plagiarism -- use to improve document
    \item \textbf{Cosine Similarity} used to measure similarity -- $\theta$ gives the degree of similarity
    \begin{equation}
        Similarity = \cos \theta = \frac{\Vec{A}\cdot \Vec{B}}{\left \| A \right \|\left \| B \right \|}
    \end{equation}
    \item \textbf{Human interpretation required because of false positives}
\end{itemize}

\section{Reference Management Tools}

\begin{itemize}
    \item Citations in our work must be according to \textbf{standard formats}
    \item Reference Management Tools \textbf{help organise and manage bibliographic resources, generates text citations and bibliography} as you write
    \item \textbf{Save bibliography from online databases} and can \textbf{switch between formats} easily
    \item Install plugin for integrating tool with word processor, and browser extension for importing references to tool
\end{itemize}

\section{Scientific Misconduct}

\textit{Violation of scholarly conduct and ethical behaviour in publication of scientific research}

Scientific misconduct does not include errors, differences in interpretations, scholarly disagreements, opinions, authorship controversies.

\subsection{Forms of Misconduct}

\subsubsection{Falsification}

\begin{itemize}
    \item Manipulation of research material, processes, or alteration or omission of observed results in an experiment
    \item Most common (over 40\% cases of misconduct)
    \item Used to improve results or remove results contradicting hypothesis
\end{itemize}

\subsubsection{Fabrication}

\begin{itemize}
    \item Invention of data, records or results
    \item Most commonly fabricated documents are consent forms and patient diaries
\end{itemize}

\subsubsection{Plagiarism}

\begin{itemize}
    \item Copying someone's intellectual property (information or ideas) without citing source
    \item \textbf{Does not distort scientific knowledge} but is not ethical and harmful for careers
\end{itemize}

\subsection{Reasons for Scientific Misconduct}

\begin{itemize}
    \item Academic, publication and career pressure
    \item Desire for fame, higher positions or financial gain
    \item Sloppy science
    \item Inability to determine right from wrong
\end{itemize}

\subsection{Consequences of Scientific Misconduct}

\begin{itemize}
    \item Dismissal from faculty
    \item Rejection of research grants
    \item Blacklisting - from hiring, funding, publications
    \item Removal of past academic achievements
\end{itemize}

\subsection{Measures to Maintain Research Ethics}

\begin{enumerate}
    \item \textbf{Before research} -- develop research plan, submit protocols for research review, prepare with research community, agree on authorship, evaluate strength of grant
    \item \textbf{During research} -- Follow approved protocol, gain consent, regularly check data, involve research community, set standards for supervision, communicate expectations, establish an Office of Research Integrity
    \item \textbf{After Research} -- Share report, follow publication ethics like citing correct references
\end{enumerate}

\subsection{Why Research Misconduct Matters}

\begin{itemize}
    \item Difficult to recognize and prevent
    \item Undermines public trust in research
    \item Corrupts scientific records
\end{itemize}

\subsection{Hazards to Good Scientific Practice}

\begin{itemize}
    \item Pressure of expectations, evaluation, publication, competition between research groups, positions and grants
    \item Involvement in commercialization, paid opinions, media presence, ambition
    \item Careless experimentation, inadequate and insufficient analysis, testing, awareness, ignorance of errors
    \item Preconceived opinions, failure to see unsuitability of data or results, emotion-driven judgement, arrogance and ambition
\end{itemize}

\section{Intellectual Property Rights}

\subsection{Patents}

\begin{itemize}
    \item A \textbf{patent} is an \textbf{exclusive government granted monopoly right to make, use or sell an innovation for a limited area and time (20 years) by stopping others from doing the same} 
    \item Patent rights are granted by \textbf{National Patent Offices}, hence patent protection for an invention must be sought in each country
    \item A patent can be bought, sold, licensed or mortgaged
    \item \textbf{Things that cannot be patented} -- ideas, anything contrary to laws, morality, common knowledge, methods of agriculture, animals, plants, computer programs, schemes, rules, literary, artistic work
    \item \textbf{Ingredients of Patents}
    \begin{itemize}
        \item Novelty and innovation
        \item Lack of obviousness
        \item Sufficiency of description
    \end{itemize}
\end{itemize}

\subsubsection{Reason for Patent}

\begin{itemize}
    \item Right to manufacture, import or sell
    \item Ability to enjoy monopoly of invention by exclusive rights
    \item Reduce competitors in market
    \item Revenue generation via licensing, assignment, technology transfer, mergers and acquisitions
    \item Confidence for venture capitalists, investors
    \item Increase value of company and build its image
    \item Key component in business strategies, helps protect company assets, allows company to operate from a strong position
    \item Encourage public interest in invention
\end{itemize}

\subsubsection{Who can Apply for Patent}

\begin{itemize}
    \item Any person (regardless of citizenship) who is the true first inventor
    \item His assignee or legal representative
    \item Alone or jointly with another person
\end{itemize}

\subsubsection{Procedure for Obtaining a Patent}

\begin{enumerate}
    \item Patent Application
    \item Publication
    \item Examination
    \item Application in Order for Grant
    \item Publication of Grant
    \item Pre-Grant opposition
    \item Post-Grant opposition
\end{enumerate}

\subsubsection{International Applications}

PCT (Patent Cooperation Treaty) Application and Convention Application

\subsubsection{Patent Application}

\begin{itemize}
    \item \textbf{Bibliographic information} -- Date of filing, name and address, title, classification, abstract, formula, corresponding priority application or patent
    \item \textbf{Technical Information} -- State-of-the-Art, description, drawings, claims
\end{itemize}

An application consists of:

\begin{enumerate}
    \item Applicants
    \item Inventors
    \item Title
    \item Address of correspondence
    \item Priority particulars of applications
    \item Particulars for filing PCT
    \item Particulars for filing Divisional Application
    \item Particulars for filing patent addition
    \item Declaration
    \item Attachments
\end{enumerate}

\subsubsection{Term of a Patent}

20 years from date of priority, maintained by paying renewal fees every year

\subsubsection{Patent Oppositions}

\begin{enumerate}
    \item \textbf{Pre-Grant Opposition}
    \begin{itemize}
    \item \textbf{Filed by any person, after publication of patent application}
    \item Filed on claims such as -- wrongfully obtaining an invention, invention is anticipated, application does not disclose source, geographic origin, or inventive step
    \end{itemize}
    
    \item \textbf{Post-Grant Opposition}
    \begin{itemize}
        \item \textbf{Can be filed by any individual involved in, or promoting work in the same field as the patent, after patent has been granted}
        \item Same grounds as pre-grant opposition
    \end{itemize}
\end{enumerate}

\subsubsection{Grant of Patent}

\begin{itemize}
    \item The exclusive right given for a period of 20 years that prevents unauthorized use of the technology
    \item \textbf{A patent needs to be granted for it to be effective} against infringement
\end{itemize}

\subsubsection{Infringement of Patent}

\begin{itemize}
    \item Injunction (permanent or temporary)
    \item Damages to profits, seizure of business, destruction
\end{itemize}

\subsection{Copyright}

\begin{itemize}
    \item \textbf{A copyright allows people to own their creative work and prevents others from reproducing it}
    \item Gives the creator control and monopoly by protecting literary, software, musical, motion picture, soundtracks, architectural etc works
    \item It is owned by the creator, but in the course of an employment is owned by the employer
    \item \textbf{Cannot be obtained for government works, ideas, concepts, common information with no originality, works that are not fixed}
\end{itemize}

\subsection{Trademarks}

\begin{itemize}
    \item \textbf{A trademark is a symbol, image, sound, word or phrase, combination of colours or label legally registered and established that represents a company, product or business}
    \item Used to \textbf{distinguish the product from competitors}
    \item Owners of a trademark can sue for damages when infringements occur
    \item Any person claiming to be the owner of a product can apply for a trademark
    \begin{itemize}
        \item Use the {\texttrademark} symbol when application is underway (8-24 months)
        \item Use the {\textregistered} symbol once registered and registration certificate is obtained
        \item Each trademark is valid for 10 years and can be renewed
    \end{itemize}
    \item Registered by \textbf{Controller General of Patents Designs and Trademarks, Ministry of Commerce and Industry, Government of India}
\end{itemize}

\subsubsection{Reasons for Trademark}

\begin{itemize}
    \item Identify the product and source for advertisement and branding
    \item Guarantees quality
\end{itemize}

\subsubsection{Documents required for Trademark Application}

\begin{itemize}
    \item Trademark
    \item Applicant details
    \item Goods or services being registered
    \item Date of first use in India (if used prior to application)
    \item Power of attorney (needs signature of applicant)
\end{itemize}

\subsubsection{Steps to register a Trademark}

\begin{enumerate}
    \item Select and authorize Trademark agent or attorney, who conducts a search for any existing or similar trademark. 
    \item If nothing is found, an application is drafted and filed by the Trademark attorney with Trademark Office who will also send you a receipt
    \item After a few days, Original Representation Sheet of your trademark is sent by Trademark attorney to you, since it has been filed with Trademark Office
    \item It can take between 18 months to 2 years for the Trademark Office to decide whether or not to grant you the trademark based on objections
    \item After being accepted, it will be published in the Trademark journal
\end{enumerate}

\footnote{A few things in the slides have been skipped in these notes - Laws, Acts, prices for patents, governing bodies etc since they did not seem important.}

\end{document}
